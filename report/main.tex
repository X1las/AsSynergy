\documentclass{article}
\usepackage[utf8]{inputenc}
\usepackage[
    backend=biber,
    style=apa,
    citestyle=authoryear
]{biblatex}

\addbibresource{references.bib}


\title{Multi-core Synergy: A Study in performance improvements utilizing multi-core threading in python}
\author{J. Wolffrom}
\date{\today}

\begin{document} 

\maketitle
\newpage

\begin{abstract}

\end{abstract}
\newpage

\tableofcontents

\newpage
\section{Introduction}

According to Moore's Law, the number of transistors on a microchip doubles every two years, leading to an exponential increase in computing power. 
However, this trend has slowed down in recent years, as the physical limitations of silicon-based technology have been reached \parencite{Mattson2014}.
\begin{quote}
    \textit{However, reaching higher clock speeds also posed significant challenges. Increased clock speeds meant higher heat generation and power consumption, necessitating more robust cooling solutions and improved power management techniques. The quest for ever-higher clock speeds eventually reached a plateau due to these limitations, leading to a shift in CPU design principles.} 
    \parencite{mscodes} 
\end{quote}

This is a testament most experts agreen on, as seen in 

As a result, the focus has shifted from increasing clock speeds to adding more cores to a processor. This has led to the rise of multi-core processors, which contain multiple processing units on a single chip. 
These processors can execute multiple orders simultaneiously, which can lead to significant performance improvements in software applications 

\subsection{Problem Statement}

The problem statement then becomes: How can we leverage the power of asynchronous programming to improve the performance of software applications? 
\newline

This question is important because [insert why the question is important here]. The goal of this study is to provide a comprehensive analysis of the asynchronous programming landscape and identify best practices for leveraging this technology.

\subsubsection{Research Questions}

This question can be split into further sub-questions, such as:

\begin{itemize}
    \item What is asynchronous programming?
    \item What are the main challenges of asynchronous programming?
    \item What are the best practices for implementing asynchronous programming?
\end{itemize}

These questions include topics such as what different asynchronous programming models exist, their benefits and downsides as well as how to implement them in practice.

\section{Literature Review}

The literature review section provides an overview of the existing research on the topic. It summarizes the key findings from previous studies and identifies the gaps in the literature. The literature review is organized by theme, with each section focusing on a specific aspect of the topic.

\subsection{Asynchronous Programming Models}

The first theme to be explored is the different asynchronous programming models that exist. This section will provide an overview of the main models, including [insert models here]. The benefits and downsides of each model will be discussed, as well as the scenarios in which they are most appropriate.

\subsection{Challenges of Asynchronous Programming}


\newpage
\section{References}
\printbibliography

\end{document}