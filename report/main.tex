\documentclass{article}
\usepackage[utf8]{inputenc}
\usepackage[
    backend=biber,
    style=apa,
    citestyle=authoryear
]{biblatex}
\setlength{\bibitemsep}{0.5em}
\addbibresource{references.bib}
\usepackage{parskip}
\usepackage{xcolor}

\title{Multi-core Synergy: A Study of Performance Improvements Utilizing Multi-core Threading}
\author{J. Wolffrom}
\date{\today}

\begin{document}

\maketitle
\newpage

\begin{abstract}

\end{abstract}
\newpage

\tableofcontents
\newpage

\section{Introduction}

According to Moore's Law, the number of transistors on a microchip doubles every two years, leading to an exponential increase in computing power. However, this trend has slowed down in recent years, as we are approaching the limitations of silicon based technology \parencite{Mattson2014}.

\begin{quote}
    \textit{\textcolor{darkgray}{Increased clock speeds meant higher heat generation and power consumption, necessitating more robust cooling solutions and improved power management techniques. The quest for ever-higher clock speeds eventually reached a plateau due to these limitations, leading to a shift in CPU design principles.}} - \parencite{mscodes} 
\end{quote}

As a result of the processing speed ceiling, the focus has shifted from increasing the clock speed of CPUs to adding more computational units to them. This has led to the rise of multi-core processors, which contain multiple processing units on a single chip, often referred to as "cores". These cores can execute multiple orders simultaneiously, which can lead to significant performance improvements in software applications. However, a discrepancy can be found in the communities that rely on these improvements.

\begin{quote}
    \textit{\textcolor{darkgray}{While there are some PC games that love CPUs with a dozen or more cores, they’re few and far between. 
    Instead, finding an 8-core, 16-thread processor with a high clock speed and a lot of L3 Cache is going to get you further than just adding more CPU cores to the equation. }} 
    - \parencite{Thomas2025}
\end{quote}

Within the gaming communities we're seeing a trend of diminishing returns, whereas what we should be seeing is an equivalance. There is a concensus that multi-core threading is the future of software development, and it has been for several years now, even as far back as 2005 \parencite{mscodes}, so the question is what's holding us back?

\subsection{The Problem Area}

If multi-core threading is the future, then it is essential to understand why it is not delivering the expected performance improvements it has the potential to deliver. \parencite[p. 12]{Rauber2023} mentions that simulations have shown that superscalar processors with up to four functional units yield substantial benefits over the use of a single functional unit. So in theory, it would seem there is a great potential to gain from utilizing multiple cores. Some of it has been utilized according to \textcite{Thomas2025}, as we are talking about 8-cores as opposed to one, but it has taken us well over 20 years to get to this point.

As such, this study aims to address this question of why multi-code optimization is so difficult to accomplish, if and/or how it can be improved, and what the potential benefits of doing so are. In addition, we wish to bring to light the concept of multi-core threading or processing as a whole, seeing as it has slipped out of the public's eye in recent years. Now, more than ever, with the rise of machine learning and AI, we need to be able to utilize the full potential of our hardware.

\subsubsection{Research Question}

This being said, there are limits to what can be achieved in such a short study, therefore we will be focusing heavily on the theory behind multi-core computing and how to get startet with it. 

The research question that we will be addressing is then as follows:

\begin{quote}
    \textit{What are the challenges of multi-core optimization, and how can they be overcome?}
\end{quote}

This question can then be further broken down into the following sub-questions:

\begin{itemize}
    \item What does the standard architecture of a CPU look like? Is there one?
    \item What is a thread, and what is a processor?
    \item What frameworks are commonly used for multi-core threading?
    \item What are the challenges of multi-core threading, and how can they be overcome?
\end{itemize}

These questions serve as a guide for the study and will help to structure the research process, beginning with a review of the literature on the topic.

\section{Limitations}

This study is limited in scope to just a handful of programming languages, and even in that case we will just be delving into the details of Python. This is due to the fact that Python isn't necessarily the best language in terms of performance, but is one of the slowest in terms of execution time. This makes python an exceptional candidate for multi-core threading, as it stands to gain the most from it.

As such, the primary focus will be the theory behind multi-core processing in python with a few examples of how to implement it. We will be looking at how languages such as C, Java, and C\# handle multi-core processing, but we will not be going into detail on how to implement it in those languages. The focus will be strictly on comparisons for the sake of understanding key concepts like parallelism and concurrency.

In addition to the limits of programming languages, we will also be limiting the range of operating systems to just Windows and Linux. This is due to the fact that these share the same range of common chipsets as opposed to MacOS, which keeps to a very small range of chipsets. This is not to say that MacOS is not capable of multi-core processing, but rather that most of the practical applications of multi-core processing are applied to Windows and Linux systems.

\section{Threads and Processors}
\subsection{Synchrony and Asynchrony}

Presentation on the two paradigms, and how they both are used in multi-core threading.

\subsubsection{Subroutines and Coroutines}

Small introduction to Subroutines and Coroutines as methods, and linking them directly to "def" in python.

\subsubsection{Suspension}

Include something about time.sleep in python, it suspends a thread just like async.

\section{Parallelism}
\subsection{Degree of Parallelism}
\subsubsection{Parallel Execution Time (PET)}

Consists of the time it takes for all cores or processors to finish a given program.

Also the time for data exchange or synchronization.

Should be smaller than the time for a synchronized single-core process to be worth it.

Influenced by idle times.

Smallest execution time occurs generally when workload is distributed equally amongst cores/processors.

Speedup and efficiency are used to quantitatively measure the parallel execution time.

\subsubsection{Load Balancing}

When the workload of a program is distributed equally amongst the machine's cores or processors.

\subsubsection{Idle Time}

The time a processor cannot do anything useful but wait for more work.

\section{Memory}
\subsection{Shared Memory}

Memory organization where the machine shares memory for all threads.

Synchronization plays a heavy role, for example by keeping threads from reading files before another has written to them.

Often connected to the term "Thread".

\subsection{Memory Access Time (MAT)}

Add content if applicable.

\section{Schedulers}

Add content if applicable.

\section{TRIN Model}

Add content if applicable.

\section{State of the Field}
\subsection{Python}

As it stands, Python is one of the most popular programming languages in the field of computer science. It is widely used in a variety of applications due to its simplicity and ease of use. However, it is an interpreted language, which means that it is not as fast as compiled languages like C or C++. In fact it is written in C, which is a compiled language. This means that in places it keeps the speed of C by pre-compiling certain commands, but in other places it is slower than C, with up to 10 times the execution in difference.

This is important due to the way Python handles multi-core threading. By default, Python comes pre-installed with threading and asynchronous libraries, which allow for the use of multiple threads in a program, however the presence of the Global Interpreter Lock (GIL) means that only one core will be in use at any time. This means that the performance improvements of multi-core threading are not easily accessed, but by using libraries such as multiprocessing, we can bypass the GIL using some C utilities. This allows us to use multiple cores in a program, but it is not as easy to implement as it is in other languages.

\subsection{Java}

Java comes with a built-in threading library, which allows for the use of multiple threads in a program, managed by the Java Virtual Machine (JVM). Usually this is accomplished by inheritance, designating a class as a thread, allowing it to use common threading methods such as sleep (suspension) and join (synchronization). This means that Java is able to utilize multiple cores in a program by use of their own scheduler, and the performance improvements are easily accessible. However, this is not without its drawbacks, as the JVM is an interpreted language, which means that it is not as fast as compiled languages like C or C++. In addition, the JVM is not as efficient as other languages when it comes to memory management, which can lead to performance issues in large programs. 

\subsection{C\#}

Include thoughts about other programming languages.

\subsection{C}

Include thoughts about other programming languages.

\section{Data Collection}

Section on how the data was collected, what was collected and the code used. 

Potentially rename the section to something more fitting.

\section{Results}

Add content if applicable.

\section{Discussion}

Include thoughts about other programming languages.

\section{Conclusion}

Add content if applicable.

\newpage
\section{References}
\printbibliography
% Ensure Biber is run to resolve undefined references

\end{document}